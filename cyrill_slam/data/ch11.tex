% !Tex encoding = UTF-8
\documentclass[12pt, a4paper]{memoir}
% \usepackage[utf8]{inputenc}
\usepackage[T1]{fontenc}
\usepackage{hyperref}
\usepackage{amsmath}
\usepackage{enumitem}
% \usepackage{clrscode3e} % CLRS 3/e package for writing pseudocode
\usepackage{graphics}
\usepackage{xcolor}
\definecolor{shadecolor}{gray}{0.9}
% \usepackage{geometry} % 设置页面布局

\graphicspath{{./images/}}
\setlength{\parindent}{24pt}

\title{Notes for Probabilistic Robotics}
\author{Bayes Nie}
\date{\today}

\begin{document}

\maketitle

\chapter{The GraphSLAM Algorithm}
\section{Miscellaneous notes}
\begin{enumerate}[leftmargin=*]
    \item \textbf{Soft Constraints}: In mathematical optimization, constrained optimization (in some contexts called constraint optimization) is the process of optimizing an objective function with respect to some variables in the presence of constraints on those variables. The objective function is either a cost function or energy function, which is to be minimized, or a reward function or utility function, which is to be maximized. Constraints can be either hard constraints, which set conditions for the variables that are required to be satisfied, or soft constraints, which have some variable values that are penalized in the objective function if, and based on the extent that, the conditions on the variables are not satisfied.
    \item \textbf{Residual Uncertainty}: The uncertainty that remains after the best possible analysis has been undertaken is what we call residual uncertainty.
    \item \textbf{Exact Inference \& Approximate Inference}: Exact inference is carried out with the posterior probability of the parameters. However, we often do not have access to that posterior — it may be difficult to compute, sample, or both! In those cases, if we can find a — necessarily biased, but simpler — approximation to the posterior, we can use that approximation to carry out inference. That’s why the inference is approximate in this case. 
    \item \textbf{Isomorphism}: In mathematics, an isomorphism is a mapping between two structures of the same type that can be reversed by an inverse mapping. 
    \item \textbf{Assignment problem}: The assignment problem is a fundamental combinatorial optimization problem. In its most general form, the problem is as follows: 
    \begin{shaded}
        The problem instance has a number of agents and a number of tasks. Any agent can be assigned to perform any task, incurring some cost that may vary depending on the agent-task assignment. It is required to perform as many tasks as possible by assigning at most one agent to each task and at most one task to each agent, in such a way that the total cost of the assignment is minimized.
    \end{shaded}
    Alternatively, describing the problem using graph theory: 
    \begin{shaded}
        The assignment problem consists of finding, in a weighted bipartite graph, a matching of a given size, in which the sum of weights of the edges is a minimum.
    \end{shaded}
    
    
\end{enumerate}


\end{document}