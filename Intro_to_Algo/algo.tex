\documentclass{report}
\usepackage{enumitem}
\usepackage{hyperref}
\usepackage{clrscode3e}

\title{Introduction to Algorithms}
\author{Note}

\begin{document}

\maketitle
\tableofcontents

% \textsc{} for definition 
% \textbf{} for leading words of a paragraph

\part{Foundations}
\chapter{The Role of Algorithms in Computing}
\begin{enumerate}[leftmargin=*]
    \item \textsc{Algorithms:} an \emph{algorithm} is any well defined computational procedure that takes some value, or set of values, as \emph{input} and produces some value, or set of values, as \emph{output}. 
    \item \textsc{Data Structure:} a \emph{data structure} is a way to store and organize data in order to facilitate access and modifications. 
    \item \textbf{NP-complete Problems:}
    \begin{itemize}
        \item Although no efficient algorithm for an NP-complete problem has ever been found, nobody has ever proven that an efficient algorithm for one cannot exist. 
        \item If an efficient algorithm exists for any one problem, then efficient algorithms exist for all of them. 
        \item Several NP-complete problems are similar, but not identical, to problems for which we do know of efficient algorithms. Hence a small change to the problem statement can cause a big change to the efficiency of the best known algorithm.
    \end{itemize}
\end{enumerate}

\chapter{Getting Started}
\section{Insertion sort}
\begin{enumerate}[leftmargin=*]
    \item \textbf{Insertion Sort:}
    \begin{itemize}
        \item \textbf{Pseudocode:} 
        \begin{codebox}
        \Procname{$\proc{Insertion-Sort}(A)$}
        \li \For $j \gets 2$ \To \attrib{A}{length}
                \Do 
        \li     $\id{key} \gets A[j]$ 
        \li     \Comment Insert $A[j]$ into the sorted sequence $A[1 \twodots j-1]$.
        \li     $i \gets j-1$
        \li     \While $i > 0$ and $A[i] > \id{key}$
                    \Do
        \li         $A[i+1] \gets A[i]$ 
        \li         $i \gets i-1$
                    \End
        \li     $A[i+1] = \id{key}$
                \End
        \end{codebox}
        \item \textsc{In Place}: it rearranges the nunbers within the array \emph{A}, with at most a constant nunber of them stored outside the array at any time. 
        \item \textbf{Loop invariants:}
        \begin{itemize}
            \item \textbf{Initialization:} It is true prior to the first iteration of the loop.
            \item \textbf{Maintenance:} If is true before an iteration of the loop, it remains true before the next iteration.
            \item \textbf{Termination:} When the loop terminates, the invariant gives us a useful property that helps show that the algorithms is correct.
        \end{itemize}
    \end{itemize}
\end{enumerate}

\end{document}